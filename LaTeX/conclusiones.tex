\paragraph{}
Despu\'{e}s de realizar la experiencia, se pueden obtener varias conclusiones. En la primera parte, se realiz\'{o} un estudio univariado para analizar la criminalidad en EEUU. Este estudio arroj\'{o} que en la mayor parte de las poblaciones (67\%),  hay entre 200 y 400 crimenes violentos por cada 100 mil personas, y las poblaciones en las que hay m\'{a}s de 800 cr\'{i}menes violentos por cada 100 mil habitantes, son at\'{i}picas y no representativas para la muestra.
\\ \\
En la segunda etapa, se estudi\'{o} la relaci\'{o}n de este \'{i}ndice de criminalidad con la tasa de empleo en esas poblaciones y el salario mensual promedio, con el objetivo de estimar cual era el que mayor relaci\'{o}n ten\'{i}a con la criminalidad. Para realizar el an\'{a}lisis se utilizaron herramientas de boxplot entre otras, de modo de poder observar las relaciones existentes entre ambas. Finalmente, se concluye que la tasa de empleo es el \'{i}ndice que m\'{a}s afectaba la criminalidad, a trav\'{e}s de un an\'{a}lisis de varianza ANOVA.
\\ \\
En la tercera etapa, se estudi\'{o} la relaci\'{o}n entre peso y altura de una muestra de individuos, buscando a trav\'{e}s de regresi\'{o}n lineal si esta relaci\'{o}n exist\'{i}a y se pod\'{i}a encontrar. Los resultados arrojados por el an\'{a}lisis indicaron finalmente que esta relaci\'{o}n no existe, y que las variables son completamente independientes.
Finalmente, se concluye que se lograron de buena manera los objetivos planteados, as\'{i} como que esta experiencia fue enriquecedora a la hora de entender a fondo los contenidos aprendidos en c\'{a}tedra, y aprender al mismo tiempo a usar herramientas de an\'{a}lisis estad\'{i}stico, como lo es R.
