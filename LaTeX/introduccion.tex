\paragraph{}
El objetivo de este laboratorio es, a trav\'{e}s del an\'{a}lisis univariado, bivariado y de regresi\'{o}n lineal, estudiar muestras de datos, de modo de comprobar en la pr\'{a}ctica los contenidos aprendidos en c\'{a}tedra. Se comienza por estudiar una serie de valores respecto a la criminalidad en EEUU, a trav\'{e}s de un an\'{a}lisis univariado, para posteriormente llegar a conclusiones fundamentadas y apoyadas con los resultados del an\'{a}lisis. Luego, se procede a buscar la causa de estos cr\'{i}menes violentos, a trav\'{e}s un estudio bivariado. Se relacionan las variables de tasa de empleo y de salario mensual promedio en las diferentes poblaciones, con la muestra inicial de criminalidad, para comprender mejor cual explica mejor la tasa de criminalidad. Una vez realizado el an\'{a}lisis se llega a una conclusi\'{o}n a trav\'{e}s de los resultados del estudio. Finalmente, en la tercera parte de este laboratorio, se busca determinar la relaci\'{o}n entre altura y peso de una muestra, para comprobar si a trav\'{e}s de esta relaci\'{o}n se pueden obtener valores desconocidos de una variable  partir de la otra. Se hace un estudio de regresi\'{o}n lineal, para finalmente llegar a una conclusi\'{o}n final respecto a la relaci\'{o}n entre estas variables.